\subsection{Act 3}

\subsubsection{Scene 1}

Banquo suspects that Macbeth is behind the death of Duncan due to what the
witches told him (but this also makes him hopeful that his descendants shall
truly become king). Macbeth learns that Banquo and his son will be riding out
later that day. Through soliloquy, we learn Macbeth's true thoughts about how
Banquo is a danger to him and how he feels he has destroyed his entire life so
that Banquo's descendants can carry on with the crown (Macbeth has no heir).
With this knowledge in hand, Macbeth convinces two mercenaries to attempt to
murder Fleance and Banquo that day.

\subsubsection{Scene 2}

Lady Macbeth enters a room in the palace to find Macbeth beside himself with
guilt and worry, saying how being dead is better off and more peaceful. Lady Macbeth tells him to calm down so that he doesn't look so distraught in front of the party guests later.

Macbeth goes on to tell Lady Macbeth to be wary of Banquo, but he decides not
tell her the full plan of his (and Fleance's) murder to spare her of the weight
of the knowledge.

\textit{"O, full of scorpions is my mind, dear wife!"}

\subsubsection{Scene 3}

Another murderer joins the party. They manage to slay Banquo, but Fleance
escapes. Without a light, they cannot make chase, so they go back to Macbeth to
tell him the situation.

\subsubsection{Scene 4}

One of the murderers comes to inform Macbeth that Banquo has been slain but not
Fleance. This upsets Macbeth, who begins to see illusions of Banquo's ghost
sitting at the dinner table during a feast with the nobles. Lady Macbeth is
upset by this behavior and makes attempts to bring him back to reality, but she
eventually has to bid the others away and cancel the feast. Macbeth asserts
that he will meet with the three witches again to know what will happen.

\subsubsection{Scene 5}

This is a bit of a weird scene (thought not to actually be written by
Shakespeare). Hecate, the goddess of witches apparently, scolds the witches for
cursing Macbeth without her, and she says that she will appear when Macbeth
arrives and show him illusions that will inspire his confidence in order to
create even more chaos.

\subsubsection{Scene 6}

Lennox and another noble can see through Macbeth's scheme (the deaths of Duncan and Banquo and who was blamed afterwards were very similar), also commenting on
how Macduff, in England, and Macbeth seem to be preparing for war with each
other. Macduff allies with Malcom and King Edward of England to defeat Macbeth.
Macduff also refuses a messenger from Macbeth, which acts as a subtle hint
towards war.

In this case, Macduff acts as a foil to Macbeth to show how he has been
consumed with ambition and fights for his self-preservation, while Macduff
fights to save the country (Lennox: "That, by the help of these [...], we may
again / Give to our tables meat, sleep to our nights, / Free from our feasts
and banquets bloody knives").
