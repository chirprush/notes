\subsection{Act 4}

\subsubsection{Scene 1}

The witches cast a spell right before Macbeth comes to meet them. The witches refer to Macbeth as "something wicked."

Macbeth meets with the three witches to learn more about his future. The
witches in turn call upon three apparitions to tell him three major prophecies. In order, the apparitions are:
\begin{enumerate}
    \item An Armed Head, which tells Macbeth to beware Macduff.
    \item A Bloody Child, who tells Macbeth to be "bloody, bold, and resolute" and that he will not be hurt by any child born of a woman. (Macbeth falsely makes him think that this means he will not be slain. Even still, he vows to kill Macduff)
    \item A Child Crowned, with a tree in his hand, who tells Macbeth to be proud/not worried of conspirers, for he will not be defeated until "Great Birnam Wood to high Dunsinane Hill / Shall come against him." Macbeth interprets this as being impossible, and is firmly rooted (heh) in the belief that he shall not be defeated. Note: Dunsinane Hill is the hill upon which Macbeth's castle is built.
\end{enumerate}
In short, the witches plan to lull Macbeth into a false sense of confidence
works by conveying only half/vague truths. Despite this though, Macbeth still
asks about Banquo's prophecy, and he is shown an image of eight descendants of
Banquo all being crowned king.

Macbeth then learns from Lennox that Macduff has fled to England. Macbeth rues
that he will kill Macduff's family.

\subsubsection{Scene 2}

Lady Macduff talks to Ross about how Macduff fleed to England. Lady Macduff
questions why he would do this and considers him as good as dead having
betrayed the family. Ross then advises Lady Macduff and her family to flee, but
she is hesitant. Lady Macduff then has an interesting conversation with her son
about traitors/liers which seems to have some deeper meaning.

A messenger, anonymous, enters to once again advise that the Macduff family
flee. Unfortunately, this is too late, and the murderers enter the house,
asking for the whereabouts of Macduff, exclaiming his treachery. In the
process, Macduff's son is murdered, and Lady Macduff escapes (I think? But
later we learn she is killed so).

These murderers are likely the same ones that killed Banquo (or at least, how
the play is written and the character list portray this as so).

\textit{"What, you egg?"}

\subsubsection{Scene 3}

The setting is in England, in front of King Edward's palace.

Macduff gains the trust of Malcom (they lament together about the state of
Scotland and Macduff pledges that he is not treacherous), and they band
together, vowing to defeat Macbeth.

An interesting description of King Edward is given, portraying him as some holy saint (assuming I'm reading this correctly). I suppose this is another instance of some element of contrasting with Macbeth going on. 

Ross enters the conversation, having come from Scotland, and also laments at
the sad state of affairs. During this conversation we also learn that England
gave Macduff commander Siward and ten thousand troops.

Ross then breaks the tragic news to Macduff that all of his family has been
killed. Macduff has no choice but to channel this grief into rage towards
Macbeth.
