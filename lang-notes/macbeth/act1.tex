\subsection{Act 1}

\subsubsection{Scene 1}

The three witches divinate their next meeting to be after a battle between has
ended, before the sun sets. This meeting place shall be a heath, a place with
low grass and shrubbery.

As we went over in class, the setting is that of medieval, feudal Scotland in its war against Norway. As this was written in the Elizabethan times, witches are regarded as purely evil (disciples of the devil), and it is truly a bad sign when they say your name.

A familiar (in this case, Graymalkin and Paddoc) is the spirit pet of the witch
(contract familiars and all that stuff like from fantasy).

The thunder and lightning detailed in the stage directions also adds to the atmosphere and truly conveys that something bad is going to happen.

A thane, as one can probably guess from context, is a lord (a feudal one specifically given the times).

\subsubsection{Scene 2}

The setting is a battle camp near Forres.

Both a bloody captain and Thane of Ross gives recount of Macbeth's heroic
victories, defeating Macdonwald, the Norweyan lord, and the traitorous Thane of
Cawdor. Duncan is delighted and wants to give Cawdor's title to Macbeth.

The rhetoric in this speech is rather interesting, although the purpose is
clear. This speech sets up the viewers' expectations for Macbeth (and Banquo)
to be truly heroic and brave. Kingly, even.

We also learn of two traitors, Macdonwald and the Thane of Cawdor, both whom
Macbeth defeats.

\textit{"What he hath lost, noble Macbeth hath won."}

\subsubsection{Scene 3}

Appearing with thunder again, the three witches enter the scene. The witches tell some weird story about a sailor and their wife; I assume this is related to the sacrifice/ritualistic stuff (especially because of the severed "pilot's thumb").

Macbeth and Banquo meet the three witches. These witches foretell that Macbeth
will gain the title of Thane of Cawdor and become king. They also foretell that
Banquo's sons will become king in the future.

Banquo: \textit{"And oftentimes, to win us our harm, / The instruments of darkness tell us truths..."}

Macbeth sees these premonitions as neither a bad nor good thing. In particular, he says that he shall leave it up to chance or fate with his becoming of king rather than action, although it's clear that he is really thinking a whole lot about the premonition.

As they go to meet the king, Macbeth asks for Banquo and him to talk about the witches tellings more later.

\subsubsection{Scene 4}

Duncan confirms that the original Thane of Cawdor has been executed and then
greets Macbeth and Banquo. He thanks both of them, with both of them pledging their allegiance to Duncan once again, and then pronounces that his
eldest son, Malcom (the Prince of Cumberland), will some day take the throne.

Note: This actually violates the Scottish laws of succession as Duncan did not
discuss this with the other thanes before the making the decision and actually
would be a reason to become angry with Duncan.

Macbeth sees Malcom as a sort of rival, and reveals to the reader that, deep
down, he does feel a forbidden desire for the throne.

\subsubsection{Scene 5}

Lady Macbeth receives letters from Macbeth telling of his exploits as well as
the witches' premonition. Lady Macbeth, knowing that Macbeth may be too kind or
unwilling to take matters into his own hands and shape his fate, wishes to take
matters into her own hands and push him forward. Upon hearing that Duncan is
coming to the castle later that night, she immediately begins to scheme a plot
to assassinate Duncan so that Macbeth can become ruler. Macbeth returns, and she begins speaking of her intentions.

Duncan is supposed to leave the next day after staying the night at Macbeth's castle, which let's Lady Macbeth plan her vicious attack.

\subsubsection{Scene 6}

The following characters are at Macbeth's castle.
\begin{multicols}{3}
\begin{itemize}
    \item Duncan
    \item Malcom
    \item Donalbain
    \item Banquo
    \item Lennox
    \item Macduff
    \item Ross
    \item Angus
    \item The attendants
\end{itemize}
\end{multicols}
This scene sets the stage for the banquet happening at Macbeth's castle. Currently, Duncan, Lady Macbeth are present, and (after exchanging pleasantries and such) Duncan wishes for Lady Macbeth to lead him to Macbeth.

\subsubsection{Scene 7}

Macbeth logically evaluates the act of assassinating Duncan, and comes to the
conclusion that he really should not kill him. Macbeth believes in the hand of
justice, so what he shall do here is clearly going to come back to bite him.
Furthermore, they are related (? "I am his kinsman") and also Duncan is, well,
the king. To Macbeth, Duncan is a great, quite holy man and has done nothing wrong at all. It would be completely reckless and terrible to kill the one who has given him rewards, and it would completely damage his reputation

Lady Macbeth enters and asks why Macbeth wasn't where he was supposed to be, to
which Macbeth tells her that he will not assassinate Duncan. In response to
this, Lady Macbeth becomes quite angry and only pushes further for Macbeth to
carry out the act.

Lady Macbeth tells Macbeth to not be a coward, to be a man. Lady Macbeth
asserts that they will not fail if Macbeth has no misgivings, and in the
process convey to the audience the exact plan they shall use to assassinate
him: Lady Macbeth will keep the two chamberlains company until they become very
drunk and then they shall attack Duncan in his sleep using the daggers of the two chamberlains.

Macbeth folds under the whole "be a man" thing, and after this bit of scheming, the two head back to return to their places to continue the plan.

\textit{"False face must hide what the false heart doth know."}
