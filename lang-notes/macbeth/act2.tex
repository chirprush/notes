\subsection{Act 2}

\subsubsection{Scene 1}

Banquo hands his son (Fleance) his sword, and then he talks with Macbeth, noting that he is still up at this late time of night. Banquo talks about how happy Duncan was, even giving Lady Macbeth a diamond.

Banquo brings up the words of the three witches, to which Macbeth lies that he
hasn't thought much about it, promising to discuss the three witches another
time. Banquo then leaves, leaving Macbeth to hallucinate a bloody dagger in
front of him. He takes this as a sign of guiding him to his goal (assassinating
Duncan) but it could very well be symbolic of what's to come after.

\subsubsection{Scene 2}

Macbeth murders Duncan in his sleep (Donalbain I believe was in the same room so O\_O). While Lady Macbeth is satisfied with the
job and tells him not to think too hard about it, Macbeth is still quite a bit
distraught (describing in decent detail what they said and how they acted as he
stabbed them) and paranoid over what he has done. He hears voices talking about
his murder and is in general quite uneasy about the whole affair.

Lady Macbeth takes the daggers from Macbeth (who for some reason kept
them\textemdash I suppose this goes to show how shaken Macbeth is about the
affair) and goes to place them back at the scene of the murder. Macbeth goes to wash his hands of the blood and does so with great paranoia.

The two hear a knocking at the door, and in turn they leave for their bedroom as to not be caught.

\textit{"Wake Duncan with thy knocking. I would thou couldst."}

\subsubsection{Scene 3}

The scene starts the next morning with the porter (the guarder of the door/gate), knocking on the door (there's some dramatic irony here as he talks about opening up the gates to hell but only we know that Duncan was actually slain).

Lennox and Macduff greet the porter at the door and talk about how he was
sleeping late at night due to alcohol. As Macbeth enters, though, the
conversation changes topic to how Duncan hasn't yet woken up (they talk about
how the night was rough so I guess there was some Shakespearean setting stuff
going on there also, but when Macbeth says it, it has a different meaning).
Macbeth, Lennox, and Macduff then go to wake him up, upon which Macduff sees
Duncan dead and exclaims so. Macbeth plays along during the entire period. From Macduff's dialogue we also get to know that Macbeth plunged the dagger into Duncan's head. Oof.

Once all three see the horrendous sight, they ring the bell to wake up Banquo,
Donalbain, and Malcom. Once Malcom and Donalbain arrive at the scene, they ask
who murdered him.

Immediately the blame is pinned on the servants, whom Macbeth said he had
killed. When he is inquired as to why (the real reason being to prevent them
from pleading their innocence), he says that he did so in a fit of rage and
emotion. The adults agree to convene and discuss the matter, but Malcom and
Donalbain decide to part their separate ways to avoid more conflict/stay safe.
Malcom goes to England, and Donalbain goes to Ireland; although, this does
slightly incriminate them.

\textit{"The near in blood, The nearer bloody."}

\subsubsection{Scene 4}

The setting is now outside Macbeth's castle.

Ross, Macduff, and an old man talk about how unnatural sightings were seen with
falcons, owls, and Macbeth's horses (the horses eat each other erm). This talk
also lets the reader know how Malcom and Donalbain are being accused of being a
part of the assassination of the king and that Macbeth will be crowned the new
king in Scone. Ross says he will go to Scone, while Macduff will go to Fife.

Duncan's body is taken to Colmekill, and Macbeth gets crowned in Scone.
