\subsection{Act 5}

\subsubsection{Scene 1}

The setting is in Macbeth's castle (which is in Dunsinane).

In this scene, we learn that Lady Macbeth is ailed in some way and has been
sporadically sleep-walking and maniacally writing on paper. This is reported by
the gentlewoman, her caretaker, to her doctor. The gentlewoman also reports
hearing her say mad things that she says she cannot tell the doctor, for there
would be no witness to confirm such a thing (likely something related to
murder).

Lady Macbeth begins sleep-walking (?) in front of the doctor and the
gentlewoman, with her eyes open (but "sense are shut"). Lady Macbeth goes to
wash her hands just like after Duncan was killed. Lady Macbeth is perhaps
hallucinating Macduff's blood on her hands. Lady Macbeth believes that they
already have power with little rivals and as such is so distraught at the
constant killings of everyone.

The doctor and gentlewoman hear this and are alarmed, knowing that they've
heard something they shouldn't've. The doctor claims that this disease is not
something that he can cure ("[m]ore needs she the divine than the physician").
Lady Macbeth then mentions the death of Banquo and Duncan, and then heads
straight to bed.

The doctor asks the gentlewoman to look over her.

\textit{"What, will these hands ne'er be clean? No / more o' that, my lord, no more o' that. You mar all / with this starting."}

\subsubsection{Scene 2}

In this scene, quite near Dunsinane, we listen to a conversation between some
Scottish soldiers who are rebelling against Macbeth: Menteith, Caithness,
Angus, and Lennox. The audience learns that the English army (whom the talking
soldiers are hoping to join), led by Malcom, Siward, and Macduff, is quick
approaching, and they are vying for vengeance. It is anticipated that they will
meet near Birnam Wood (the other place from the prophecy). We also learn that
Donalbain will not be there with the army, but Siward's son will be.

Macbeth, we learn, has been fortifying the castle out of rage/madness. The
soldiers talk about, with such a terrible reputation, he shall quickly lose the
weight and power of the title, and soldiers will only follow orders out of
obedience and not passion.

\subsubsection{Scene 3}

This scene depicts Macbeth in his castle, almost crazed over what the witches
told him. He is told the English army of ten thousand soldiers is advancing,
but he has no fear because of what the witches have told him. Macbeth asks
Seyton (some attendant) for his armor even though he doesn't have to fight
simply because he doesn't think that he can die (yet at least). Macbeth tells
the doctor to cure Lady Macbeth despite him saying that he can't.

\subsubsection{Scene 4}

The setting switches to the English forces and, in particular, Malcom, Siward,
Macduff, and Menteith (others are there also; these are just the people who
talked).

The English forces arrive at the Birnam Wood (the forest). Malcom orders every
soldier to take a branch (a "bough") to conceal the true number of soldiers.
This is one of the parts of the witches' prophecy being fulfilled.

\subsubsection{Scene 5}

The setting switches back to Macbeth's castle, which is being fortified and prepared for war.

Macbeth intends to wait out the siege of the English troops and quite
confidently boasts that he has forgotten the "taste of fears." A short while
later, Lady Macbeth cries out and dies (honestly this part was a little
weird?), causing Macbeth to ponder over the meaninglessness of life. A
messenger then tells him of the English advancing, holding the branches, and
Macbeth realizes this as the part of the witches' prophecy coming true.

To some regard, I believe he's accepted his death.

\subsubsection{Scene 6}

The setting switches back to the English forces, who now throw down their
branches.

Malcom delegates the first battle to be led by Siward and his song, while he
and Macduff shall do other things (he wasn't really clear on this idk).

\subsubsection{Scene 7}

The setting switches to a field during the battle.

Macbeth encounters Young Siward (Siward's son), and slays him in battle (bro literally just killed him and said "Thou wast born of woman."). Macduff, elsewhere on the battlefield, greatly wishes to fight Macbeth and take revenge for his family.

Meanwhile, Siward tells Malcom the status of the war and how it is strongly
being won by the English forces. They enter the castle.

\subsubsection{Scene 8}

Macbeth does not wish to commit suicide, even knowing that he may die, because he can still fight (I think at least? He says "Whiles I see lives, the gashes / Do better upon them.").

Macbeth meets Macduff in battle now, and Macduff is enraged by how horrible of
a person Macbeth is. Macbeth boasts that he cannot be defeated by anyone born
of a woman, but Macduff tells him that he "was from his mother's womb /
Untimely ripped." This strikes fear into Macbeth, who at first vows not to
fight him. Macduff then tells him to yield and then die, to which Macbeth has
no choice but to begin fighting.

Macduff kills Macbeth and his body is carried away. Malcom laments the deaths
of those in battle. Ross tells Siward that his son died (nobly) in battle.
Siward doesn't seem to be that sad at all (lmao he was actually quite happy
that his son died "like a man" like dang ok bro he's gone. Bro even told Malcom
not to grieve for him).

Macduff returns to Malcom with Macbeth's head and then cheers for the new King
of Scotland (Malcom). Malcom names everyone to be earls, and then talks about
recalling those who were exiled or left Scotland because of Macbeth. Malcom
intends to be crowned at Scone.
