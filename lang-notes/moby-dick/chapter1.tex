\subsection{Chapter 1}

This chapter really sets the setting for who really the author is and what the
book is about. Previously a schoolmaster, Ishmael (I assume that this is his
story in some aspect) finds the sea as a way to calm down and escape from the
stresses of daily life. Ishmael is not a captain nor a passenger, but rather a
part of the sailor crew, in which he finds no shame being. Ishmael also
describes how strangely attracted to water and the sea many people eventually
are throughout their lives.

At the end of the chapter, Ishmael finds it "his fate" to go on a long sailing
journey (alone?), describing the beauty of the great whales that he might
encounter.

Even from a light skim, we can see that Ishmael (or at least the man in the
supposed story) is quite religious, making allusions back to mainly Christian
events, but also is in general familiar with ancient mythology. Ishmael makes
mention of ancient Greece with the story Narcissus, ancient Egypt, and even the
Persian empire among other things.

As a side, note Ishmael also makes mention of Patagonia and in the excerpts
there does seem to some mention to some well-known landmarks from Patagonia,
which leads me to believe that, given its beautiful sights and wild landscape,
much of the journey will take place there. This is pretty cool because we
watched an interesting documentary on Patagonia in Spanish class, so I'm hoping
I'll have a better understanding of the setting.

One major question I hope to answer along the course of reading this book is:
\textbf{Why was this book chosen for AP Lang reading?} In other words, what
message does Mr. Rhinehart (or whoever chose this really) want to convey to us?
Sure, it could be very likely that there is no reason other than it provides a
rigorous enough level of reading and analysis or simply that it is a classic
reading that is common for our grade level, but I really do think there is some
better reason as to why this book in particular was selected, and I hope to
uncover this as the book continues on.

One interesting subsection I saw that could possibly relate to this is: "But these
are all landsmen; of week days pent up in lath and plaster---tied to counters,
nailed to benches, clinched to desks" (Melville 14). Ishmael describes the
difference between the "land-dwellers," tied to their homes and contrasts them
with the free sea-loving people like him. In a way, we could consider this an
analogy to the modern day with those caught up in technology, social media, and
other stressors instead of just enjoying life for it is and adventuring. Yes,
this does make me sound like a boomer, but it's a valid comparison >:)

Another good question to keep in mind is: \textbf{Why was the title "Moby Dick"
chosen?} I am aware that there is a chapter called "Moby Dick," so perhaps this
will be revealed then, but it doesn't hurt to have this in the back of our
minds.

Lastly, perhaps a more standard question that one might see as a prompt:
\textbf{How does the first person perspective of Moby Dick impact the reader's
experience?}
