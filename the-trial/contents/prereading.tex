Franz Kafka is acclaimed to be one of the greatest and most creative literary
minds of the 20th century (even having the term "kafkaesque" coined after him),
so I feel it almost necessary to dive into what is said about his writing style
and to get familiar with it before reading the book.

Kafka is said to have an \Comment{absurdist and existentialist}{One would think
these to perhaps be at odds with each other, which is interesting.} writing style.
Kafka deals very skillfully with the concept of existential anxiety, prompting
along a very piercing question: who can we look at and point to, \textbf{who
can we find fault in when no singular individual has done anything wrong?} In
this way, Kafka calls out the needless and suffocating bureaucracy of the world
(indeed, there have been comparisons made to modern day insurance companies).

It must be noted, though, that the modern image of Kafka as a mysterious,
depressing figure is very far off. The usage of absurdism is in a way comical,
a dark humor of sorts, and Kafka was known for reading his writings in front of
friends and laughing alongside them. Kafka distorts and takes to the extreme
the problems, dreads, and dreams of his own as a form of relief.

Kafka also explores the ideas of loneliness and identity, in part a projection of his own life experience. It will be interesting to see what of it still rings true in modern day society.

Ultimately, what follows will be a collection of annotations of the novel:
summary, commentary, and further analysis. Looking back at my progress this
year in terms of note-taking and analysis, I am slightly disappointed. While I
am doing just fine in AP Lit from a grades standpoint, I am not here because of
a grade; I am here to expand my intellectual ability in literature. I can only
take away from this class what put into it, and so I desire to use this time to
take genuinely proper notes.

Without further digression, we proceed.
