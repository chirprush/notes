\documentclass[a4paper, 12pt]{article}

\PassOptionsToPackage{colorlinks=false, hidelinks}{hyperref}
\usepackage{chirpstyle}
% \usepackage[colorlinks=false, hidelinks]{hyperref}
\usepackage{soul}

\geometry{top=3cm, rmargin=7cm, lmargin=1cm}

\newcounter{fixcounter}
\newcounter{tipcounter}
\newcounter{reviewcounter}

\setcounter{fixcounter}{0}
\setcounter{tipcounter}{0}
\setcounter{reviewcounter}{0}

\newcommand{\hlc}[2]{{\sethlcolor{#1}\hl{#2}}}
\newcommand{\fix}[2]{{\addtocounter{fixcounter}{1}\hyperref[fix\arabic{fixcounter}]{\hlc{WildStrawberry}{#1}}\hypertarget{backfix\arabic{fixcounter}}{}\marginpar{\footnotesize \label{fix\arabic{fixcounter}}\hyperlink{backfix\arabic{fixcounter}}{\( \textcolor{WildStrawberry}{\blacksquare} \)} \color{black} \textbf{\arabic{enumi}.} #2}}}
\newcommand{\tip}[2]{{\addtocounter{tipcounter}{1}\hyperref[tip\arabic{tipcounter}]{\hlc{Green}{#1}}\hypertarget{backtip\arabic{tipcounter}}{}\marginpar{\footnotesize \label{tip\arabic{tipcounter}}\hyperlink{backtip\arabic{tipcounter}}{\( \textcolor{Green}{\blacksquare} \)} \color{black} \textbf{\arabic{enumi}.} #2}}}
\newcommand{\review}[2]{{\addtocounter{reviewcounter}{1}\hyperref[review\arabic{reviewcounter}]{\hlc{Cerulean}{#1}}\hypertarget{backreview\arabic{reviewcounter}}{}\marginpar{\footnotesize \label{review\arabic{reviewcounter}}\hyperlink{backreview\arabic{reviewcounter}}{\( \textcolor{Cerulean}{\blacksquare} \)} \color{black} \textbf{\arabic{enumi}.} #2}}}

\setlength{\marginparwidth}{6cm}

\begin{document}

\section*{MLK ``Letter From a Birmingham Jail'' Analysis}

\textbf{Audience}: The audience, as laid out by King in the first paragraph (and the author’s note) is, in particular, eight Alabama clergymen who criticized King’s approach to fighting injustice in Birmingham, although they believe in the same ideals as King.

\review{\textbf{Purpose}}{Make sure to check this later because the full purpose may not realize itself until I've read everything in depth.}: King’s goal in writing the letter is for it to answer and defend against the doubts of the clergymen who disagree with his course of \fix{action}{Perhaps I need more context, but what even is his "course of action"? Be more specific.}, but he wishes to do so peacefully, almost negotiating.

\textbf{Exigence}:

\textbf{Example Paragraph 1 (Rhinehart)}:
Martin Luther King begins his response to his critics (fellow clergymen) with a respectful tone acknowledging that he has heard them as peers. He compliments them by elevating them to a group who warrants his immediate attention and response given that he normally does not take the time to do so. In effect, King disarms his audience via an approach of sincerity and mutual understanding. He avoids a defensive posturing which would in no way help his purpose and his need to be heard regarding his presence in Birmingham. King establishes a basis for dialogue with an audience that while morally in alignment with him has been critical of his approach. In essence his rhetorical structure results in a response that avoids conflict and suggests cooperation among colleagues who morally agree but find themselves in need of communication. In short, His critics remain open to his position on the subject.

\tip{\textbf{Analysis}}{What is he saying, why, what is the effect, what bigger picture does this lead to? \par Try to look at the interconnections between adjacent paragraphs and look at the macro more. It's quite easy to get swept up in looking at the smaller-scale rhetoric and the picture it draws, but the macro is especially important to consider how everything contributes to the overall argument and purpose.}:
\vspace{-0.3cm}
\begin{enumerate}[topsep=0pt]
    \item In the first paragraph, King aims to address his audience, the critical clergymen, \fix{succinctly}{Change this word probably.} and in a respectful manner, as opposed to being defensive. Through directly talking to his audience and mentioning that he usually does not do so for others, King exhibits genuine respect for his critics and \fix{\_}{Finish the thought}. King states foremostly that the letter is not written from the position of defense but rather of \fix{\_}{Finish the thought}, and in particular, he states that they are of the same fundamental beliefs. Through this, King attempts to build comradery with his critics instead of trying to dissect and attack their points.
    \item \tip{Through this paragraph}{Try not to summarize. The goal here is analysis, not interpreting the direct wording of the letter so that others can understand it (to this end, I don't think the math writing helps here). While analysis is birthed out of the understanding of the content, they are certainly not one and the same. \par To help this previous point, I need to try and act as if I were a critic to generate arguments and see how these are addressed by King. If I can do this, I will have ample content with which I can write about.}, King gives justification for his being in Birmingham. Principally, King dismisses any notion of being an outsider, as he believes the critics may acclaim him to be, by noting that he was invited to Birmingham by a trusted group from Alabama: the Alabama Christian Movement for Human Rights. King gives valid, concrete justification for his being in Alabama, which sets up his next point in the following paragraph. \fix{\_}{I should probably expand more on each of the paragraphs I write, but I think it would be better to read and analyze all the paragraphs first while noting down the really major details.}
    \item King gives a more abstract but fundamental reason for being in Birmingham, one that speaks even more to his character. That is, King frames himself as a moral combatant against injustice. \fix{\_}{There is far more to elaborate about this point, but I'm hungry and I can't think well right now so do this later}. Moreover, King gives allusion and even directly compares himself to a holy apostle, bolstering the fact that he is for good and the people of Birmingham. Only considering this paragraph standing alone, the clergymen may not have considered this a valid justification, but the previous paragraph and this paragraph together give King undeniable moral reason to be in Birmingham and lead his nonviolent direct-action program.
    \item With this paragraph, King not only further refutes the idea that he is an outsider to Alabama, but he goes beyond this and states that no one can truly be an outsider given the interconnection between all states. By doing this, King contributes directly to the idea that he and the clergymen are unified in an effort to fight injustice, not enemies. King wants to aid and protect Birmingham, the home of his critics, not go against it. \fix{\_}{There is certainly more I can build off of this that isn't like directly stated in the text. Think deeply of the implications}.
    \item King starts off the paragraph by directly addressing the main complaint of the clergymen, that is, that they dislike the demonstrations being given. In doing this, King brings focus to their main argument and sets up to disassemble it throughout the residing and following paragraphs. King foremostly points out that the opposing argument is incomplete and does not address the causes of why the demonstrations came about, only the aftereffects. Furthering his tone of respect, however, he does not attack his critics for this, but states that they are unified in the opinion that having nothing done would be harmful for Birmingham. To end the paragraph, King lends an image of an unfair, prejudiced society as the only other alternative to the actions of his group, further justifying Birmingham’s need for the demonstrations.
    \item In this paragraph, King takes a rather concrete, factual approach to argue the necessity of his campaign’s presence in Birmingham. King points out injustice on a spectrum of issues such as segregation, brutality, and lack of communication given by officials, asserting that they are undeniable instances of brutal injustice. King reframes the arguments of his critics so that denying the demonstrations given by his campaign would be tantamount to denying the existence of injustice in the city, a completely justified truth.
    \item To further argue on the side of undeniable truth, King gives an anecdote of how the campaign agreed to stop demonstrations temporarily in exchange for the removal of racist signs, and as a result, they were betrayed by the merchants who posed the offer. By doing so, King not only reinforces the cruelty of the merchants in taking advantage of the nonviolence activists, but it also shows that they were willing to and even did end up negotiating with those in Birmingham. The reason why negotiations did not end up working was not because of the campaign’s side, but the \fix{\_}{Finish the thought.}
    \item King builds upon this anecdote by explaining that it is not the aberration, but the norm. The society of King's time, especially Birmingham, is built upon disappointment and betrayal of his people, and with this, he quite strongly shows that he cannot be complacent and act as his critics wish. \fix{\_}{I'm pretty sure something can be said about the self-purification process, I just have no clue what to put that really has any depth to it.}. King, being \review{fully transparent to his critics}{I can probably even talk about this transparency more.}, also lays out the exact reasoning behind the planned timing of the direct action campaign.
    \item This paragraph feeds in from the previous one, in which King shows the group's desire to abide by the rules and uphold proper democratic proceedings with the election. In telling this, King demonstrates that the central motivation for the direct action is to bring attention and not the kind of injustice it is trying to fight against. \fix{\_}{I lost my train of thought on this but there is something to write about how this defuses the argument of the critics}. King once again references the all-to-familiar cycle of waiting for change, just as his critics say, to be met with nothing. It is with this simple instance of King implementing what his critics have been continuously saying and being met with failure that completely rules out any such simple arguments. It is not because his people are impatient, violent, or wishing to defy trouble that they are engaging in direct action; it is because they are left with no other choice after waiting so long.
    \item It is with this paragraph that King truly reveals the real reason for introducing his story at Birmingham previously. \fix{\_}{This shouldn't be hard to fill in and transition, I just need to actually write stuff down.}. King pries apart the negative, destructive connotation of tension from a dually, otherwise neglected constructive viewpoint for it. This type of subversion is something that King later uses again in his letter, and it almost parallels the very outcome King hopes to bring to Birmingham.
    \item By explaining that the end goal of the direct action programs is to motivate negotiation, King rebrands the criticism given by the clergymen to be exactly what is pushing for. It is with this and the unifying tone used in "our beloved South" for example that King fully shows his rhetorical strategy of not fighting back but instead trying to unify his critics and him under the same basic ideals. Indeed, King's nonviolence policy is reflected through his rhetorical strategy. \fix{\_}{Perhaps add something more about the monologue versus dialogue line? I feel like there's something but idk.}
    \item King goes on to address why the program did not give the new city administration time to act, rebutting another one of the critics' points. King once again emphasizes that action is needed in order for the government to act, once again relying on time and time again past experience to support this. \fix{\_}{This entire paragraph is kinda bad, so fix it and add more substance.}
    \item With this paragraph, King puts down the notion of the program's actions being not well-timed by arguing that it is an entirely selfish and out of touch view, effectively denouncing those that argue the point to be as such in turn. It is through describing the \review{emotions}{This is good, but it may need more depth.} that lay behind the simple phrase "Wait." that King brings the letter in the viewpoint of him and his fellow people. This sets up for the next paragraph.
    \item In transitioning to the viewpoint of those oppressed in Birmingham, King delivers a emotionally powerful paragraph full of grief and suffering that puts the audience \review{as close as they can}{I'd like to put in the idea that this is only but a sliver of what it must be like.} to experiencing what segregation and oppression feel like in Birmingham. This completely shuts off any sort of argument from the clergymen that King and his program are impatient in any way. \review{The clergymen are not like King}{I'm not sure whether this style of writing is legit for analysis (perhaps I'm overthinking this too much, but it's not very much in the style of Rhinehart's paragraph so idk). Try and revise this later.} and have not experienced what King and others have, so they could not possibly argue for prolonging their dehumanizing suffering, especially not if they agree morally with King's cause.
    \item \fix{In continuing to dismantle the arguments of the clergymen}{Remember to introduce this notion in the earlier paragraphs when he actually starts pruning all the arguments made.}, King addresses the seeming contradiction at play between following some laws and not others. Just as how King pulled apart the meaning of tension, he does so again with the law. In breaking apart moral law from immoral law, the effects are twofold. Not only does King give proper justification for his civil disobedience, addressing the critics' point, he also further emphasizes how he is on the side of good moral, with his program's aim to correct these unjust laws. \review{King is also quick to point out that he is not the first to believe this, as religious figure St. Augustine also said this in the past.}{I don't know if this is useless maybe if I explain more why this is important it should be fine to leave included.} This reference also sets up for the \tip{next paragraph}{While it's good to make connections, try to vary transitions and make them more meaningful}.
    \item To prevent ambiguity in his split notion of moral law and to remove any potential for counterargument, King concretely lays out what determines a law to be moral or not through this paragraph and those following. It is again through references to respectable figures such as Thomas Aquinas (a clergyman himself) that King once again shows this is not a new, radical concept but rather quite the opposite; it is an already established, prominent notion. King's argument is then straightforward. \review{If immoral laws are not to be followed, and segregation is immoral, for which King cites that it is a sin, then surely it is morally just to campaign against it and disobey it.}{This feels too mathy and I'm not sure if this is kosher analysis.} Through this logic, King uncontestably justifies the morality of his actions and his reason to not follow these laws.
    \item King goes on to reinforce his notion of just and unjust laws further. In speaking of laws created by a majority (by power) that only act on a minority, King directly calls out segregation, once again showing it to be a law not morally correct to follow. \fix{\_}{I'm not sure how much more I can write about this paragraph, but check over this later I suppose.}
    \item King goes even further with his argument against the fairness of the laws he disobeys, questioning the underlying fundamentals of how the law was passed. If King's critics truly believe in the American democratic system, they must come to terms with the fact that the unjust laws detailed by King were not democratic in any way. King, in effect, argues that the means by which the unjust laws were passed were illegal in the very first place, only further calling to question the reason to follow the laws. \review{If one believes in the government or law, they must acknowledge that the unjust laws outlined are falsely based.}{This sentence seems like fluff; man it's kinda hard to tell whether I'm writing anything that's actually worthwhile.}
    \item King also admits that the interpretation and execution of some just laws can make them unjust. This only further motivates King's civil obedience and shows the injustice lying in the law system. \fix{\_}{Is there really anything else I can write here?}
    \item In describing the injustice prevalent through the law system, King frames his approach as the most reasonable one. The morally correct option, King argues, is not to fully abide by the entire law, which is partially unjust, and it is not to reject the entire law, for that would reduce Birmingham to anarchy. The most optimal solution is to abide by the just laws and reject the unjust laws, precisely what King is doing.
    \item King, using a myriad of historical examples, once again states that he is no way radical in his arguments. \review{If one were to argue with his approach, they certainly cannot attack its logic, and it would go against not just King's but the beliefs of also many important or religious figures.}{This just feels weird.}
    \item Continuing from the previous paragraph, King explains how the concept of legality has been construed and separated from morality countless times in the past, all the more supporting his point that laws are not an end all be all, and that he would come to this same conclusion regardless. Notably, though, King draws a powerful comparison between the situation of African Americans in the US and the Jewish people in Nazi Germany. This comparison of oppression is one which connotes much grief and suffering, and even further demonizes segregation and those would try to prolong it.
    \item King summarizes some of his reasoning while making his statement out to those he claims are the biggest problem: the moderate whites who prefer inaction. To these people, King expresses great disappointment towards these people, but following his initial rhetoric of commanding a respecting, unifying tone, he does not fully berate them. Instead, King subtly uses this as a call to action, \review{requesting}{I'm kinda bordering on this word choice.} that those with the power to make change, such as the white moderate, do so, allow for the direct-action programs to progress, and understand King's point of view. \fix{\_}{There's definitely more to add here.}
\end{enumerate}

\textbf{Revising}:
\vspace{-0.3cm}
\begin{itemize}[topsep=0pt]
    \item[\( \textcolor{WildStrawberry}{\blacksquare} \)]  \arabic{fixcounter} object(s) to fix.
    \item[\( \textcolor{Green}{\blacksquare} \)]  \arabic{tipcounter} tip(s).
    \item[\( \textcolor{Cerulean}{\blacksquare} \)]  \arabic{reviewcounter} object(s) to review.
\end{itemize}

\end{document}
