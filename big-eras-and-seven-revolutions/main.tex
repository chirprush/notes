% Created 2022-08-10 Wed 15:29
% Intended LaTeX compiler: pdflatex
\documentclass[11pt]{article}
\usepackage[utf8]{inputenc}
\usepackage[T1]{fontenc}
\usepackage{graphicx}
\usepackage{grffile}
\usepackage{longtable}
\usepackage{wrapfig}
\usepackage{rotating}
\usepackage[normalem]{ulem}
\usepackage{amsmath}
\usepackage{textcomp}
\usepackage{amssymb}
\usepackage{capt-of}
\usepackage{hyperref}
\usepackage[margin=1in]{geometry}
\author{Rushil Surti}
\date{\today}
\title{Notes: The Big Eras of History}
\hypersetup{
 pdfauthor={Rushil Surti},
 pdftitle={Notes: The Big Eras of History},
 pdfkeywords={},
 pdfsubject={},
 pdfcreator={Emacs 26.3 (Org mode 9.1.9)}, 
 pdflang={English}}
\begin{document}

\maketitle
\tableofcontents

\newpage

\section{Big Eras}
\label{sec:org7f245bd}
\subsection{Humans in the Universe}
\label{sec:org50b08ef}

\textbf{Time Period}: 13 Billion - 200,000 years ago

\subsubsection{Overview}
\label{sec:org971152e}

This Big Era details the creation of the universe and the various events that occurred leading up to the evolution of our human species, the Homo sapiens.

This section will help to tackle three important questions:

\begin{itemize}
\item What were the origins of our ancestors?
\item Where do we stand in the universe?
\item What is our significance?
\end{itemize}

\subsubsection{Why?}
\label{sec:org5b97996}

\emph{Why should we learn about the creation of us and our universe?}

Our knowledge of our creation and history changes how we perceive us and our own history. Stories of creation also form the basis of many religions and cultural values, important influences on the lives of many people all throughout time.

\subsubsection{Creation in Religion}
\label{sec:org314de88}

One popular example of a creation myth in religion is the one found in the Bible.

The Bible states that God creates the universe in seven days, starting with light and dark, the environment and Earth itself, then plants and animals, and finally humans.

This is not entirely unlike the reasoning that modern science gives today. Earth was created first, with living organisms following suit, although the time tables are slightly scaled.

\subsubsection{The Early Universe}
\label{sec:orgabddf0f}

Scientists theorize that the universe started with an event called the Big Bang around 13.7 billion years ago. Nothing is known about what, if anything, existed before this.

The Big Bang is characterized by a sudden burst and materializing of energy and matter. The universe expanded at a rate faster than the speed of light but shortly slowed to the speed of expansion nowadays. At this stage, the universe was a hot mass of pure energy and subatomic particles.

Eventually, the universe cooled enough for protons and electrons to come together to form hydrogen and helium. Over time, large clouds of these gases would accumulate and fold in on themselves due to gravity. This caused the formation of stars. More stars accumulated and formed galaxies.

As stars grew bigger, they created new elements through fission and fusion. Once these stars died out, they introduced these new elements into the universe.

\textbf{Note}: All complex matter in the universe was formed at one point from stardust. This is but one reason why planets and life form near stars.

\subsubsection{The Solar System}
\label{sec:orgf6b4891}

Around 4.5 billion years ago, in a specific point far from the center of the Milky Way galaxy, gases and heavy metals collected to form the solar system. While the sun took up most of the matter, small parts clumped together at certain distances due to gravity. These caused the formation of our planets. The heavy metals sunk to create the cores of planets, while the lighter elements formed the crust. The gases contributed to the atmosphere.

\subsubsection{Formation of the Earth}
\label{sec:orgb78d380}

The Earth started as a hot ball of mass, constantly being hit with asteroids and other space material. After four billion years, however, the Earth cooled down, and the water vapor (thought to come from the asteroids) condensed and formed oceans.

As the Earth cooled, different layers of the planet were formed: the core, mantle, and crust. Along with these came the formation of tectonic plates, which contributed to the ever-changing surface of the Earth.

These tectonic plates are responsible for the formation of mountains, volcanoes, islands, and various features of the Earth, majorly influencing the people and cultures that would soon come.

\subsubsection{Early Life on Earth}
\label{sec:org58088b6}

Not much is concretely known about the first origins of life, but it most probably started with underwater volcanoes forming complex chemicals and creating a "primordial soup" in which life could be born out of. The first organisms were single-celled and evolved gradually.

Some populations of cells evolved the ability to produce energy from the sun through photosynthesis. This had the side effect of producing oxygen. This oxygen slowly built up, transforming the Earth's atmosphere.

Other, more complicated cells called eukaryotic cells also formed. Eventually multi-cellular organisms also came into being. Gradually, plants, animals, and other organisms found their way onto land.

\subsubsection{Human Ancestors}
\label{sec:orgddc80f4}

After the extinction of the dinosaurs, a certain group of species called mammals flourished. In particular, one group, the tree-residing primates, was quite well off. Primates had better vision and comparatively large brains. Hominins, our ancestors, were a part of the primates and could stand up.

\textbf{Note}: Hominins appeared in Africa. This means that all humans originate at some point from Africa.

Other hominin species such as the \emph{Homo erectus} and Neanderthals branched off to other parts of the world.

\textbf{Question}: Why did our ancestors appear in Africa first and not elsewhere? Other primates were able to survive the climate of other parts of the world, so that was likely not the determining factor. At this point, farming had not been invented yet either. Was there simply an abundance of resources? Were there less predators?

Hominins were intelligent creatures, having social hierarchies, being able to use tools, and utilizing fire.

\subsubsection{Differences with Humans}
\label{sec:orgc3f5cf2}

Hominins changed very little in the two million years that they were around. In this time, they did not develop religions, villages, or art.

\emph{So, what makes us different that allowed us to do all of these things?} This is the primary question that Big Era 2 answers.

\textbf{Prediction}: Cooperation and collaboration are key in realizing many of the previously listed achievements. At the cornerstone of these is \textbf{communication}. I believe that the difference that sets apart hominins from Homo sapiens is language, or some form of communication that powers the sharing of ideas more complex than simple actions.

\subsubsection{Summary}
\label{sec:orgcb7540d}

The universe started with a bang. After a while of cooling and passing of stars, complex matter was formed. Complex matter accumulated in places to form stars and planets, including Earth. Life on Earth started with single-cellular organisms created by chemicals sourced from underwater volcanoes. These organisms evolved and life diversified, leading to plants and animals. Fast-forward millions of years, primates are flourishing and a certain group, called the hominins, is able to migrate, stand up, form complex social hierarchies, and use tools. Despite this, they are not as smart or progressed as humans.
\subsection{Human Beings Almost Everywhere}
\label{sec:orga8eda67}

\textbf{Time Period}: 200,000 - 10,000 years ago

\subsubsection{Differences}
\label{sec:org454db9b}

The primary difference between the second Big Era and the first Big Era is the appearance of Homo sapiens, or humans. This era marks the start of human history.

The primary difference between this era and the next one, Big Era 3, is that farming had not been invented yet. This era is classified as the Old Stone Age, or Paleolithic period. The next era is classified as the New Stone Age, or Neolithic (hint: Neo) period.

\subsubsection{Overview}
\label{sec:orgc4ee58c}

This era contains without a doubt the larger period of human history.

Much of the evidence and reasoning that supports the assertions about this era come from archaeological evidence and genetic mutations due to a lack of any written items or history.

\subsubsection{Why Archaeology?}
\label{sec:orgbbcbbe7}

\emph{Why and how would you use archaeology as evidence for historical research?}

Bones, tools, food, and other items can help scholars uncover the way people lived, their diet, the development of technology, the climate at the time, and countless other details. Along with carbon dating, a timeline of information can be developed. This can give us a rather accurate look into the life of the early humans.

\subsubsection{Why DNA?}
\label{sec:org0179482}

\emph{How can the analysis of DNA aid in historical research?}

By taking genetic sequences of modern humans and comparing them to each other, we can pick out shared and unique mutations that can tell us more about the history of early humans. People coming from the same ancestors will share most of the same mutations. If a population was isolated from that group of people, modern-day descendants will not share those mutations. This gives us a way to see when groups separated from each other and perhaps where they migrated to.

\subsubsection{Art and Language}
\label{sec:orgf5d5796}

Art, found in caves and other surfaces, is the most accurate indicator of how early humans perceived the world. Art gives historians a better look into possible religions or other more abstract, complex ideas that they were trying to convey. Art also partially implies the existence of language.

Language and complex communication as a whole is the crowning achievement of the Homo sapien race that distinguishes it from all other animals. Humans are able to communicate abstract concepts and emotions. This allowed for the accumulation of knowledge in communities, called "collective learning."

\subsubsection{Technology}
\label{sec:org101bdbc}

With the development of language and sharing of ideas, the rate of advancement of technology began to increase. Tools became more specialized, intricate, and varied.

\subsubsection{Migration}
\label{sec:org957783f}

As technology evolved, many groups migrated eastward, and the population of Homo sapiens became more interspersed.

It is thought that humans first left Africa and traveled somehow, perhaps by land or through the Arabian Sea by boat, to India. From there, the early explorers ventured further to the east, reaching Southeast Asia and China. A benefit of the journey to Eurasia was that the constant tropical climate would allow for much more abundant resources and progress.

\textbf{Question}: What motivated the early humans to travel across such distances? Was it intentional? Was there overcrowding or a scarcity of resources?

Some groups made their way to Australia and New Zealand by a connecting piece of land. Others ventured to Russia and even Siberia. The harsh environment in those areas warranted the creation of new technology and shift to different diets and hunting techniques.

Through another connecting landmass, some humans reached North America, spreading from Canada to South America.

\subsubsection{Environmental Impact}
\label{sec:org98a484a}

Humans quickly found hunting quite easy in their new found environments with animals not used to them. This caused them to over-hunt, leading to the extinction of many popular species, including the mammoth. Humans also burned plants, contributing to new plant growth.

Surprisingly, the takeover of humans also led to the extinction of their relatives, the Neanderthals and hominins, who did not have the complex communication and technology of the humans.

Humans now inhabited the globe. Although the population grew into the millions, the relative size of groups remained the same, only the number of groups having increased.

\subsubsection{Early Society}
\label{sec:org5f09215}

Not much is known about the specific details of how society and culture worked in the age before writing. We can however gain some vague clues from modern hunting-and-gathering societies.

In these communities, meat is valuable, and hunters and gathers are both knowledgeable. These groups are nomadic and travel as families. Social roles are based on age and gender, and no one accumulates wealth because there is no reason for it.

Nearing the end of the era, communities began to settle down. In order to secure food, they started to take care of the local plant life. This would lead to farming later on.

\subsubsection{Art}
\label{sec:org3ef4cff}

Forager communities often painted depictions of animals and nature with some sorts of spirits. As well, jewelry and colorful shells were at the center of artistic expression at the time. Figurines were created by people living in regions without caves. Music was also developed with several woodwind instruments.

Unfortunately, it is hard to determine the true meaning of such objects without context, so multiple viewpoints may apply.

Despite these many art forms, no sort of written language was developed.

\subsubsection{Summary}
\label{sec:org978af7e}

Homo sapiens, or humans, emerge from the hominins and have better brains capable of complex thoughts and communication between each other. Technology rapidly develops and the humans migrate, conquering the globe. While taking over the globe, they have massive impacts on their environments. All the while, an artistic explosion is happening with paintings, jewelry, and music developing. As people begin to settle down at the end of the era, techniques similar to farming start to develop. A written language still has not been created.
\subsection{Farming and the Emergence of Complex Societies}
\label{sec:org71e27d1}

\textbf{Time Period}: 10,000 - 1000 BCE

\subsubsection{Overview}
\label{sec:org327321f}

In this era, we see the steep rise of the invention of farming and the effect it has on the environment and societies. Farming and the drastic social changes it brought distinguish this Big Era from the previous two.

This era is called the neolithic era, or New Stone Age, due to the innovation and invention of new stone tools, particularly those related to farming.

\subsubsection{Farming}
\label{sec:org7c4914a}

The actual developing of farming was a relatively slow process and did not appear everywhere in the world at the same time. Compared to the timeline of previous human events, however, farming spread quite quickly

Historians define farming as activities that increase the production of useful items, such as cattle and crops, and the reduction of useless items, like pests and weeds. Farming is not limited to just food. By farming, humans change their environment to suit their needs. This developed agrarian societies, in which communities revolve around agriculture.

Farming used a process called domestication, in which humans selectively breed plants to change them to their benefit. As plants were further domesticated, they became co-dependent, unable to produce without human intervention. This also had the slightly unfortunate side effect of preventing a society from returning to foraging in that area.

\subsubsection{Population Change}
\label{sec:orgcc5618c}

As opposed to extensification (the increasing of communities) seen in the previous chapter, this era, in large part due to farming and communities settling down, sees intensification (the increasing of the populations of communities). This caused the population to sky-rocket into the hundred millions, also causing social changes in communities.

\subsubsection{Climate Change}
\label{sec:org76f5205}

In this era, the globe gradually started to warm up. This led to the shrinking of glaciers and rising of sea levels. This had two major effects.

\begin{enumerate}
\item Land bridges connecting Afroeurasia, the Americas, and Australia disappeared underwater, disconnecting travel.
\item Parts of the Northern Hemisphere warmed up and became much more bountiful, particularly the Fertile Crescent, an area in the Middle East.
\end{enumerate}

\subsubsection{Social Change}
\label{sec:orga11b828}

Particularly in places such as the near Tigris River, Euphrates River, Nile River, the Indus Valley, and Huang River, populations flourished and communities evolved to be more complex cities and civilizations.

Cities did suffer consequences, however. Cities were particularly vulnerable to change in their environment, disease and socioeconomic and political problems.

In the beginning of the era, most people were still on equal status. Everyone farmed, and while there were leaders, they did not have power over people.

After a few thousand years, however, complex societies formed. This led to a wide variety of advancements

\begin{itemize}
\item Cities formed.
\item Jobs became more varied and specialized.
\item Social hierarchies and patriarchy developed.
\item A centralized government was formed.
\item Trades took place.
\item Technology advanced.
\item Architecture was built.
\item A writing or record system was made.
\item Religion, laws, and art progressed.
\item Sciences and mathematics were worked on.
\end{itemize}

In places were agricultural was not viable, such as the Sahara Desert, the system of pastoral nomadism was implemented. These were societies based on herding domesticated cattle, living off of their products while migrating.

These two types of societies interacted, sometimes peacefully and sometimes violently. The migrational tendencies of the pastoral nomadic tribes also helped them foster trade between cities.

\subsubsection{Change in Ideas}
\label{sec:org4a050d4}

With the condensing of populations, information and ideas were spread at a faster rate and in more abundance. Religious knowledge, technology, and writing were spread quite quickly.

One civilization in particular, the Sumerians of Mesopotamia, made countless breakthroughs in various areas.

\begin{itemize}
\item A system of writing, cuneiform, was developed.
\item Mathematics in base-10 and base-60 was developed.
\item Government, commerce, astronomy, and architecture improved.
\item Concepts such as bronze metallurgy and the wheel were invented.
\end{itemize}

Each civilization had their own dynamic cultural styles.

\subsubsection{Summary}
\label{sec:orgdba0589}

The newest invention on the block has, albeit gradually, come: agriculture. The domestication of plants and animals to the needs of humans coupled with global warming allows people to settle down and populations of communities themselves to grow. This leads to complex communities and civilizations forming, containing governments and even social classes. These complex communities lead to more ideas, technology, and innovation, including the appearance of an early writing system.
\subsection{Expanding Networks of Exchange and Encounter}
\label{sec:org04677af}

\textbf{Time Period}: 1200 BCE - 500 CE

\subsubsection{Overview}
\label{sec:org5983b37}

This Big Era is all about the increasing of networks and communication between the newly formed cities and states from Big Era 3. In this Big Era, roads and sea networks emerge, connecting large parts of the world together and transmitting goods, technology, and ideas (particularly religion).

\subsubsection{Environment}
\label{sec:org4b9d33b}

In this period, novel changes in the environment were not seen; however, agriculture and its impact on the environment did increase.

One example of this is deforestation. People needed more area for farming and living, leading to mass deforestation in some parts of the world. This had quite a few consequences. Wood shortages, famines, and the potential extinction of species were all shortcomings of the mass clearing of land. This might have even contributed to the changing of the planet's climate.

\subsubsection{Technology}
\label{sec:org4279c87}

Techniques such as breeding hybrid crops and using horses and camels for work began to appear. In addition, this era saw the emergence of iron metallurgy. With the increase of networking, technologies such as these rapidly spread across the world. Despite this, some areas that were not as connected, such as the Americas and Oceania, were not as quick to pick up on them.

\subsubsection{Populations}
\label{sec:org78b9c08}

Population and its rate of growth continued to rise. This was due to agriculture as well as built up immunities to common diseases. Because of the crowded nature of cities, these immunities were built up naturally and allowed for more population growth.

Once again, this population growth warranted a restructuring of the layout of societies, causing many to move towards cities. Despite this, though, the amount of cities declined at the end of the era, likely due to pandemics and the declining of empires.

Two particularly important, large cities were Rome and Luoyang. These cities, like others, served as centers for government, trade, art, and religion among other things.

\subsubsection{Empires}
\label{sec:orgdb996c7}

With faster ways of transmitting information (horses, roads, etc.), leaders could exert more power over larger areas of land. This aided in the rise of empires. The three main empires at the time were the Han empire, the Persian empire, and the Roman empire. \textbf{These states were named empires because one governmental body exercised control over different, diverse groups of people.}

From around 100 CE onward, several empires, almost spanning the entire length of Eurasia, prospered. This led them to exchange products and ideas between each other. This led to increased trade activity on the silk roads.

\subsubsection{Society}
\label{sec:orgc18ce7c}

Inside the cities, novel social changes took place. Social classes became more varied and based off of wealth and power. There was also an increase in the usage of slaves. Finally, societies moved further towards patriarchy.

\subsubsection{Language}
\label{sec:orgabf8d7d}

One important distinction between this Big Era and the previous one is the development of new writing systems. These writing systems used alphabets instead of logographic characters. This allowed for an increased literacy rate. Some logographic systems still remained, however.

\subsubsection{Religion}
\label{sec:org28fe336}

As a consequence of an easier way of writing and a more efficient way of transmitting this writing, major religions developed. Six major religions at the time were:

\begin{enumerate}
\item Buddhism
\item Christianity
\item Confucianism
\item Daoism
\item Hinduism
\item Judaism
\end{enumerate}

With the exception of Islam, these make up the modern major religions. These all likely were formed due to the need of a moral system at a bigger scale. These religions all have their own characteristics and ways of answering existential questions about human life. Some of them, particularly Christianity, had different branches and variations.

Particularly in Greece, philosophy emerged, with scholars trying to find reason and patterns in nature.

\subsubsection{Summary}
\label{sec:org7339670}

Following the big boom of agriculture, the population keeps growing and cities follow suit. Communication and interaction between these cities and states develop and becomes critical for the exchanging of goods, technology, religion, and ideas. With this efficient communication comes empires, now annexing more land than ever and reaching people of varying ethnicities. The alphabet develops, leading to higher literacy. Finally, major religions and moral systems are formed with the spread of ideas.
\subsection{Patterns of Interregional Unity}
\label{sec:orgebf904a}

\textbf{Time Period}: 300 - 1500 CE

\subsubsection{Overview}
\label{sec:org70b04bf}

This Big Era marks the end of the ancient world and the emergence of some modern world traits. In this period, the growth rate again climbed, more empires and states were formed, trade (especially in the Indian Ocean) skyrocketed, and major religions further developed.

\subsubsection{Population}
\label{sec:org6e67f91}

At the beginning of the era, the population actually declined for around three centuries, likely due to drier conditions, epidemics, and economic recession.

After a while, the growth rate climbed back and continued for seven centuries. Nearing the end of the era, another recession occurred after several epidemics, including the Black Death, hit Eurasia. By the end of the era, the population again recovered and reached around 400 million.

\subsubsection{Agriculture}
\label{sec:org2cab768}

One major reason for the increase of population at the end of the era was the improved agricultural technology. Technology enabled more land to be made into farmland and increased the output per area of land.

\subsubsection{Trade}
\label{sec:org7818f65}

New technology allowed groups to move at a faster pace on both land and water. This not only empowered empires militarily, but also greatly promoted trade. In particular, new vessels and navigational tools allowed for water to serve as another route for trade, with the Indian Ocean being the most active. This is also another reason for the growth of the population.

Towards the middle of the era, trade and commerce picked up in Afroeurasia. In particular, China, India, and Islamic states became wealthy through manufacturing goods and trading them. This wide network of trade soon got nearly the entire eastern hemisphere involved.

\subsubsection{Environment}
\label{sec:org5ae7288}

The environment takes a pretty heavy hit in this Big Era. With the demand for wood and urbanization increasing, deforestation rates also increased. As a consequence, soil degraded, floods became more common, famines occurred, and wood became more scarce.

\subsubsection{Empires}
\label{sec:org42e8c5c}

With new technology, pastoral nomads posed a big threat to many established empires. Invasions from the Uighurs, Huns, Arabs, and others contributed to the fall of the Roman empire.

New empires were built, such as the Gupta empire of India, Arab empire of the west Afroeurasia, and Tang and Sung empires of China. After Mongol invasions, however, these empires too collapsed. The Mongols took over their land and built the largest empire in history, spanning 7 million miles.

In Europe, the Byzantine and Russian empires appeared. In Africa, the states of Ghana and Mali formed. In Asia, the Srivijaya and Majapahit empires formed. Several other empires formed in the Americas.

\subsubsection{Technology}
\label{sec:orgeb91d6f}

With new technology in the form of catapults, bows, and even gunpowder muskets, warfare became more bloody and complex.

In terms of weapons, naval warfare did not change much, but navigation and the distance ships could travel did improve. This also allowed for Christopher Columbus in the late 15th century to discover the Americas.

Land and sea trade routes allowed for the exchange of new inventions, technology, and ideas. This Big Era saw advancements in mathematics, language, astronomy, and education itself. The abacus, paper, and the printing press were developed and subsequently spread via trade routes. In Muslim and Christian society, predecessors to universities were formed as places were intellectuals could gather. The idea of a college was also introduced and spread throughout Europe.

\subsubsection{Religion}
\label{sec:orgfdd7453}

This Big Era saw the emergence and spread of major belief systems. Religions such as Hellenism (originating in Greece and focusing on reasoning), Judaism, Christianity (split between the Roman pope and the Eastern Orthodoxy), Hinduism, Buddhism, Daoism, and Confucianism all spread through various places in Afroeurasia.

One major happening in this Big Era was the formation of the religion Islam, one of the three universalist religions. Islam is a monotheistic and missionary religion, having a scripture, the Qur'an, and the Prophet Muhammad. Due to conquests and conversion, Islam quickly spread and gained large ground, stretching from Spain to India and including large parts of Northern Africa.

It is important to note that religion was more prevalent and strong near large cities. In more rural areas, local religions were practiced, most involving nature.

In the west hemisphere, indigenous religions consolidated. With the exception of the Mayans, Aztecs, and Incas, local religions were much more common.

\subsubsection{Summary}
\label{sec:orgeee626f}

With trade routes and new navigational technology, the world became more connected than ever. Technological innovations and political changes allowed for a greater population, in spite of several epidemics. Major religions developed and spread, with Islam emerging. Although there was mass deforestation and environmental damages, humanity managed to progress at a rate faster than anything before.
\subsection{The Great Global Convergence}
\label{sec:orgdc15969}

\textbf{Time Period}: 1400 - 1800 CE

\subsubsection{Overview}
\label{sec:orgd1550b5}

In this Big Era, the connections between individuals and societies further develops. The most defining event that happened in this era was the connecting of every inhabited region of the world to each other, called "The Great Global Convergence." Five major changes distinguish this era.

\begin{enumerate}
\item Human connections become more complex, notably with the new interaction between Afroeurasia and the Americas.
\item The Columbian Exchange occurred, involving the transport of plants, animals, and bacteria between the Americas and Afroeurasia. Diseases brought along caused the deaths of many indigenous people, and Europeans brought back home new crops.
\item A fully global economy was formed with the currency being silver.
\item Europe rose in military and political power.
\item Western Europe underwent the Scientific Revolution and the Enlightenment, which challenged several religions and philosophies.
\end{enumerate}

\subsubsection{Humans and the Environment}
\label{sec:org94a3399}
\begin{enumerate}
\item Population Changes
\label{sec:orga0de179}

The population grew even faster in Afroeurasia during this period. Conversely, the population plummeted in the Americas, called the Great Dying. What did remain the same was that only a few percent of people stayed in cities or were foragers; most people, 95\%, continued to lived as farmers.

\item The Great Dying
\label{sec:orgbb8b2f4}

After the Spanish arrived in the Americas, they accidentally brought along with them microorganisms and bacteria. The indigenous people had no immunities to these microorganisms, and this led to outbreaks and pandemics that ravaged the population.

This affected both the indigenous and Spanish people. The outbreak brought about the fall of the Aztec and Inca empires. In addition, decreased populations of Native Americans also resulted in labor shortages for the Spanish.

The Spanish also brought along domesticated animals and plants. Because the climate was similar to Europe's and the existing wildlife had not adapted to them, the new wildlife quickly took over and transformed the environment of North America.

\item The African Slave Trade
\label{sec:orgd8d08db}

With the need for new labor in the Americas, especially for farming, Europeans brought in African slaves in large numbers. The slaves proved to be a profitable solution for the new settlers. Around 11 million able African workers were taken from their homes.

Although they still remained in the minority in terms of population, around 2 million Europeans migrated to the Americas.

\item Rampant Deforestation
\label{sec:orgbb6ad81}

Due to higher demands for wood as well as the expansion of mining, deforestation plagued large parts of the world.

In response, Europe moved towards other solutions, eventually settling on fossil fuels. Japan, in contrast, worked towards reforestation.
\end{enumerate}

\subsubsection{Humans and Other Humans}
\label{sec:orgb23955d}
\begin{enumerate}
\item A Shift in the Economic Center of Gravity
\label{sec:org6e0af17}

Although Asia was at the center of trade and commerce during the start of this Big Era, Europe quickly closed the gap with the precious metals and sugar they were able to obtain in the Americas. This benefited both sides.

\item Revolution in Military Power and Finance
\label{sec:org3951fd4}

Europe also rose in power due to innovations in gunpowder weaponry, strategy, tactics, and other elements. Europe fought multiple battles during this period, allowing them to innovate in warfare technology.

Not every country was equal, however. Especially in North America and South Asia, Britain prevailed over France.

Finance and economy had to be reconsidered by many countries in order to fund the large amounts of weaponry. Britain ultimately won over France due to France's lesser funding, part of which went to the American Revolution.

Although the power was shifting to Europe, most countries were still not able to take on countries in Asia and Africa that were also armed.
\end{enumerate}

\subsubsection{Humans and Ideas}
\label{sec:org7f78c80}
\begin{enumerate}
\item Cultural Developments in Europe
\label{sec:orgec91056}

Following the long period of plagues, Europe revived itself with the Renaissance, a cultural expression of new wealth. The Renaissance by itself, however, was not a true, major cultural change. Much more changing was the Scientific Revolution of the 17th century.

One invention, the printing press, was invented in the 15th century and allowed ideas to be spread much faster. This had major implications on cultural and religious developments. An example of this is the Protestant Revolution, led by Martin Luther and causing multiple branches to form from the original Roman Catholic Church.

\item The Global Religious Scene
\label{sec:orga8876c7}

Christianity and Islam quickly spread through the Americas and Afroeurasia respectively. Despite these developments, no religion held a monopoly on the world, and most followers were not very strict or knowledgeable.

\item The Scientific Revolution
\label{sec:org547ed00}

With Christianity being challenged with the Protestant Revolution, discovery of the Americas, and knowledge from other parts of the world, many scholars saw the chance to develop reasoning and science to question the old way of thinking. These scholars argued that the world was not entirely governed by religion, but it worked based on natural laws.

This would lead to further developments in the Enlightenment in the 18th century.
\end{enumerate}

\subsubsection{Summary}
\label{sec:org338457e}

Western Europe pioneered a conquest to travel west and discovered the Americas, forever changing the history of the planet. The Europeans quickly took over large spans of land and extracted resources such as precious metals and sugar, at the expense of African slaves and the lives of many indigenous people. This, coupled with wars, allowed Europe to gain a more powerful economic and militaristic standing in the world. Back in Europe, the Renaissance and Protestant Revolution shook the continent, challenging the authority of the Church. Reasoning and scientific thought developed during the Scientific Revolution, which also questioned the Church's prevalence over society.
\subsection{Industrialization and its Consequences}
\label{sec:orgd82697e}

\textbf{Time Period}: 1750 - 1914 CE

\subsubsection{Overview}
\label{sec:orgb1bbd81}

This Big Era was an important era for history and is often called the era of the "modern revolution." The events of this era are termed "autocatalytic" because changes that occurred in this era caused other changes to happen, leading to exponential growth and accelerating change.

This Big Era featured six major related developments.

\begin{enumerate}
\item Humans now moved on to use fossil fuels and steam for power.
\item Massive population growth happened, with the population reaching more than one billion people by the end of the era.
\item The Industrial Revolution occurred.
\item Transport and communication advanced.
\item A movement for democracy emerged.
\item Empires gained power.
\end{enumerate}

\subsubsection{Humans and the Environment}
\label{sec:orgcfb13c1}
\begin{enumerate}
\item Fossil Fuels
\label{sec:org29ca8db}

Fossil fuels such as coal, oil, and natural gas allowed for greater energy output at a lower cost. These factors led to coal providing 90\% of all energy for humans by the end of the era.

\item Many More People
\label{sec:orgb01757c}

By the end of Big Era 7, the world population had an increase of 200\%, with 1.79 billion people on the planet. This, combined with fossil fuels, greatly influenced the environment.

The population quickly grew in western Europe, the same thing later happening in Asia and Latin America.

Due to epidemics, the indigenous population declined.

With more people, urbanization increased and cities grew bigger. A much larger population lived in cities compared to earlier Big Eras.

\item Mass Migrations
\label{sec:org9a79f89}

With steamships and railroads, large volume migrations became much more reasonable. Around 50 million people from Europe migrated to places in the Americas and Australia. These settlers pushed out indigenous people and exploited the resources of the land.

1.7 million Africans were forcefully migrated to the Americas to work as slaves.

In Asia, around 30-40 million Indians and 15 million Chinese migrated to find work in mines and plantations in Europe, the Americas, and Africa. Some stayed, while others left due to poor treatment.

Internal migrations were also common in countries.

\item Environmental Impact of Industrialization and Migration
\label{sec:org85d56a3}

With new technology, humans had more power to change the environment. Deforestation and pollution increased. Plants and animals from parts the world were transported to environments that had not adapted to them. This led to the spread of invasive species.

In addition, multiple significant famines took place, caused by a shortage of food and ignorance from country leaders. These famines showed the gap in living standards between developed and developing countries.
\end{enumerate}

\subsubsection{Humans and Other Humans}
\label{sec:orga32587e}
\begin{enumerate}
\item The Emergence of Liberalism
\label{sec:orgab5de09}

Along with the modern revolution came the ideas of liberalism, which favored individual rights, free markets, republican governments, and change. Liberalists called for private property, better communications and transport, modern markets, democracy, and separation of church and state.

\item Economic Trends
\label{sec:org3030f44}

With coal and steam power, production increased. The invention of railroads allowed for the faster transport of humans and goods.

These progressions allowed for a change in the world economy. Rural countries adapted to export raw materials and import finished goods from more industrial countries. An example of this is cotton, which overtook sugar as the most important commercial crop and become an important import for Britain, which exported cotton textiles.

Sometimes, European countries went against the belief of a free market. An example of this is the Opium War between Great Britain and China. Britain fought a war to force China to open trade for Britain's opium. This strained the Chinese economy and led to an outbreak opium addictions. Several other countries including the United States used this policy in trading with Asian, Middle Eastern, and Latin American countries.

Gold was standardized following the discovery of it in various countries.

The end of this Big Era saw another explosion of growth, with new technology such as the telephone and the semi-automation of agriculture.

Following liberalism, companies, holding private capital and property, grew in power.

The downside of the inclusion of a global economy was that it potentially made individual economies weaker.

\item Political Trends
\label{sec:orgfccec31}

At the start of the Big Era, Britain acted as the top global power.

Around the world, several revolutions occurred with the ideals of liberalism at heart. The Thirteen Colonies of America was the first revolution, followed by the French Revolution, Haitian Revolution, and other revolutions in Latin America. These revolutions replaced their governments with republics. Voting rights were given to adult males who held property. At first, slavery was not dealt with, but anti-slavery movements appeared later on.

Another major change was the abolishment of slavery. Although Britain and America abolished the slave trade, it was not until near the end of the Big Era that slavery was completely eliminated. Despite the defeat of slavery, other forms of involuntary work existed.

\item Colonial Encounters
\label{sec:orga4d51fd}

The gap between the rich and poor was drastically widened with the expansion of European colonies.

There was resistance to take over from many local residents, but they could not win against the advanced weaponry of the Europeans. At the end of the era, European countries controlled most of the world.
\end{enumerate}

\subsubsection{Humans and Ideas}
\label{sec:org45f2211}
\begin{enumerate}
\item The Rise of Secularism
\label{sec:org4c03bf2}

Following the liberalist idea of progress and change, people used science as a way to challenge traditionalist religions. The secular, or worldly, side argued that events did not necessarily occur with purpose and that the world was not created supernaturally. These ideals spread throughout Europe and sparked much debate between both sides.

\item 19th Century Racism
\label{sec:org69691f8}

Many people construed Charles Darwin's theory of "survival of the fittest" to justify racism and colonization.

\item Thriving Religions
\label{sec:org0318a23}

Even with secularism and science challenging religion, faith spread and grew at a faster rate. Religious scriptures could be written and sent across the globe. In particular, Christian missionaries were sent all around the world in order to spread their faith.

Religion, being more traditional, was opposed to fast change, liberalism, and colonialism. Despite these beliefs, none could challenge the weaponry and power of the European countries.

As liberalism spread throughout the world, many debated economic reforms. In the end, they were not popular with the elites or those who suffered from change.

The viewpoints on both slavery, popular sovereignty, and workers' rights changed drastically, but positively, from the start to the end of the era.

Another major change founded in this Big Era was the development of the idea of nationalism, which led to these major revolutions.
\end{enumerate}

\subsubsection{Summary}
\label{sec:org2c48a6f}

This Big Era is deemed the era of "modern revolution," with the large changes it brought. Industrialization, with fossil fuels and steam at the center, greatly advanced the world's technology and led to even more population growth as well as massive changes to global economy. Along with industrialization came colonialism and liberalism. With colonialism, many countries in Europe began conquests and took over land across the globe to create colonies. Liberalism, on the other hand, pushed for the creation of republics, called for fast change, encouraged revolts, challenged religion, and completely changed society and politics.
\subsection{A Half Century of Crisis}
\label{sec:org07f4741}

\textbf{Time Period}: 1900 - 1950 CE

\subsubsection{Overview}
\label{sec:org4b6c86a}

As Europe industrialized, it slowly began to take over and influence the world. Countries that were not industrialized often lost economic standing in the world, unable to keep up with automation and technology of other countries.

Increased competition and rapid change led to six major happenings that changed the world as a whole:

\begin{itemize}
\item The natural environment was strained to due fast growth.
\item Costly tariffs harmed equal global economics.
\item Two world wars greatly harmed European and certain Asian countries, both economically and physically.
\item Countries with rising economies challenged Europe's monopoly of the world.
\item Nationalism and colonialism lost influence and weakened.
\item New world events caused the re-evaluation of liberalism and elitism.

By the end of era, two superpowers emerged, the United States and the USSR.
\end{itemize}

\subsubsection{Humans and the Environment}
\label{sec:org186de77}
\begin{enumerate}
\item New Technology and Population Growth
\label{sec:orgbbd0f80}

Despite wars and conflict, the population continued to grow in all parts of the world. With new technology like fertilizers, food production increased.

Better understanding of medicine and diseases allowed for the creation of vaccines, which greatly increased life expectancies and led to a decline in the death rate.

\item The Shift to Cities
\label{sec:orgd0b4a1d}

Rapid population growth caused many people to move from rural parts of countries to cities and other countries, causing urbanization. Sanitation in cities led to even lower death rates.

\item Humans Reshaping the Environment
\label{sec:org91e1b7e}

Both cities and farmland encroached onto previously untouched land. This led to even more deforestation, particularly in tropical areas. This deforestation and over-farming led to land erosion. One severe consequence of this was the United States dust bowl.

People gradually began to use energy and produce more waste. Coal was used even more and oil was used in combustion engines over steam, polluting the air.

The GDP of the world doubled and production increased greatly. In addition, nuclear energy was also discovered. These factors combined showed that humans were the single most influential force in the environment, signaling the beginning of the Anthropocene era.
\end{enumerate}

\subsubsection{Humans and Other Humans}
\label{sec:org864fb89}
\begin{enumerate}
\item The Economic Roller
\label{sec:org61dc28a}

Between 1870 and 1950, the global economy increased, decreased, increased, and decreased again, leaving the economy worse than it was in 1870.

This happened due to countries becoming protectionist and imposing tariffs. These increased competition in global markets. After the World War I, which settled some of the competition, surplus goods still did not sell well. War costs and German and Austrian reparations complicated finance systems and led to many loans, especially from American bankers.

After Americans bankers pulled their money back, the world economy crashed, causing production and employment to fall. This "Crash of 1929" hit industrialized regions and the regions that depended on them. As governments spent less and raised more tariffs, the situation worsened.

It was only after the government started spending money, stimulating the economy with the New Deal, that the Great Depression started to improve. This led to a higher involvement of government in economies.

The last push that ended the Great Depression was the beginning of World War II.

\item The Great War
\label{sec:orgd387f37}

As countries began making weapons and forming alliances in Europe, tensions rose. After the assassination of Archduke Francis Ferdinand of Austria-Hungry, a chain reaction occurred, starting the first World War.

World War I proved to be violent, killing millions. Back at home, governments were able to gain more power and the lives of citizens were transformed.

After multiple years of fighting, the United States joined the Allied side, which was the last push that ended the war and made Germany surrender.

The war left tens of millions dead or injured and destroyed economies and infrastructure.

The Treaty of Versailles left Germany and Austria to face harsh reparations. In addition, the first world body, the League of Nations, was formed in order to prevent future wars, but it held little power.

\item Soviet Union
\label{sec:orgcbd6a96}

Russia saw the creation of the first communist state, which went in direct opposition of the capitalist system. After economic strains because of the war, Tsar Nicholas II abdicated. In his place came the Bolshevik Party, which was headed by Vladimir Lenin, a Marxist who was against liberalism.

Under the Bolsheviks, Russia left the war, won a inner struggle, and got rid of influential capitalists. This transformed Russia into the USSR, which was poorer and less advanced than its earlier counterpart.

The successor of Lenin, Joseph Stalin, industrialized Russia, changed its economy, and took away land from the peasants.

At the cost of famines, labor camps, and the lives of millions, the USSR achieved a powerful economy. This went on to inspire China to become Communist under Mao Zedong.

\item Fascist Governments
\label{sec:org94cea76}

Germany's Adolf Hitler pioneered Nazism (National Socialism) and advocated for fascism. Hitler quickly rose to power with his Nazi party and became the leader of Germany. As a result, Germany disregarded the Treaty of Versailles and prepared for war and invasion. Fascism was also implemented in Italy by Benito Mussolini.

\item Nationalism in the Colonized World
\label{sec:org21f5ce6}

Inspired by liberalism and nationalism, colonies began to challenge their European rulers.

After losing the war, Germany and Austria lost their colonies to Britain and France. Turkey, on the other hand, became an independent state.

Nationalist movements empowered other vies for independence from countries such as India, countries in Africa, China, and Vietnam.

\item Challenges to Democracies
\label{sec:org43b5158}

Democracies were pressured to make voting more available to populations such as women, with New Zealand being the first to do this.

\item World War II
\label{sec:org120f957}

Tensions and challenges in the 1920s and 1930s eventually bubbled up to start World War II. Japan invaded China, Italy invaded Ethiopia, and Germany invaded Austria, Czechoslovakia, and Poland, which signified the start of the war.

After allying with the Soviet Union to invade Poland, Germany turned on them. Japan attacked Pearl Harbor, bringing the United States into the war. As a result, World War II was fought on every continent in the world except Antarctica.

Eventually, the Allies overpowered Germany and Italy, and the US used two nuclear weapons on Japan, which ended the war entirely.

Around 60 million people died, and expenses were even higher. People also saw the true horror of how Jewish people were being treated during the Holocaust.
\end{enumerate}

\subsubsection{Humans and Ideas}
\label{sec:orgb723e22}
\begin{enumerate}
\item Science and Art
\label{sec:org12d743d}

This era saw scientists and artists such as Albert Einstein, Werner Heisenberg, Sigmund Freud, and Pablo Picasso. New genres of art emerged and motion pictures, radio, and jazz brought a new form of entertainment.

\item Mass Communication and Popular Culture
\label{sec:org1a6dbf5}

Popular culture spread to the entire world. Movies and radios allowed people to reach far audiences and transmit novel ideas and cultures. Newspapers and sports developed, and luxury and entertainment no longer became exclusive to the upper class.

By the end of Big Era 8, the world was no longer controlled by Europe and no longer fixed on liberalism.

Many wondered whether peace would be achievable after the Industrial Revolution.
\end{enumerate}

\subsubsection{Summary}
\label{sec:org13796bd}

As two World Wars break out, major social, political, and economic changes occur. Liberalism and rapid change is no longer seen as a goal by many, but nationalism continues to be a driving factor that challenges colonialism. The political concepts of communism and fascism emerge, with Russia and China being the main proponents of communism.
\subsection{Paradoxes of Global Acceleration}
\label{sec:orgde1ad74}

\textbf{Time Period}: 1945 - Present

\subsubsection{Overview}
\label{sec:org46977c4}

One unique point about this Big Era is that it is still underway. Many things are still unknown, but there are some trends and a few important events.

\begin{itemize}
\item The human population has almost reached 8 billion people. With many living in cities.
\item Environmental harm and strain has became even more prevalent and worse than ever.
\item Humans have developed better technology to extract energy.
\item The Cold War occurred, the United States became the leading global power, and the United Nations was formed.
\item Modern technology like the internet and mobile phones allowed for even greater connection with people all over the world.
\item Weaponry and country management became expensive, leading to the straining of the economies of many states. In addition, terrorist groups have arisen.
\end{itemize}

This era is described as paradoxical, with many people experiencing both extremes of wealth, living conditions, and opportunities.

\subsubsection{Humans and the Environment}
\label{sec:org85379ab}
\begin{enumerate}
\item Population Growth and its Environmental Effects
\label{sec:org5441141}

Average life expectancies rose from 35 to 55 years old, causing an increase in population. This was due to new medicines, sanitation, and healthcare.

Agriculture also developed with irrigation, fertilizers, pesticides, and GMOs. This did, however, weaken the food chain.

Population growth caused an even greater impact on nature than before, with many animals going nearly extinct.

\item Colossal Energy Consumption and the Environment
\label{sec:org87a4da5}

Worldwide energy consumption increased by 400\%, and reliance of fossil fuels went up.

Large clearings of land were made for automobiles, and more efficient travel also meant more efficient disease spread. Many chemicals, such as the ozone-layer-destroying CFCs, were created.

The use of fossil fuels put large amounts of \(CO_2\) into the atmosphere, which contributed to global warming and other drastic, harmful forms of climate change. Nuclear power also proved to be somewhat dangerous due to the radioactive nuclear waste.

Some nations have started plans to repair the environment and stop diseases, but not all nations were able to.

Birth rates slowed down in more urban regions.
\end{enumerate}

\subsubsection{Humans and Other Humans}
\label{sec:orga530662}
\begin{enumerate}
\item Big Science
\label{sec:org38ec80d}

Governments became more involved in research and development. This led to the discovery of DNA and inventions in biotechnology.

\item Global Migration
\label{sec:orgc6630b0}

Cheaper travel and relaxed movement restrictions led to millions of migrants. Many moved to North America and Australia, and others moved to bordering countries.

\item Electronic Communication
\label{sec:orgf72f205}

Phones and computers quickly became common household items. This allowed for the world to become globally connected and share culture in a much more efficient way.

\item Postwar Economic Growth and Trade
\label{sec:org6b4e73c}

After the war, the Allies invested money into reviving the world economy and trade, which contributed to the GDP growing immensely. More workers also appeared, further helping the progress.

In wealthier nations, living standards increased, widening the gap with developing countries. As developing parts of the world accrued debts, the global growth slowed.

\item The Cold War and its End
\label{sec:orgea21adc}

The Cold War was a period of raised tensions between countries holding two different economic ideologies, capitalism and communism. The Cold War also resulted in many colonies achieving independence.

After the Cold War, the Soviet Union broke into several states, returning the world economy to following mostly capitalist ideals.

The eastern hemisphere gradually grew as well, with the formation of the European Union and Russian Federation.

\item Sovereignty and Rights
\label{sec:org7bd7a97}

One major development of this era was nationalism-powered anti-colonialism.

This big era also saw the standardization of global human rights with from the UN. Because of this, more attention was brought to children's rights and forced labor other than slavery.

\item The Passing of Peasantries
\label{sec:org86dbb49}

In most parts of the world, less and less people worked as peasants on land. More than half the population lived and worked in cities.

\item Women's Rights
\label{sec:org12de14f}

Following the formation of the UN, a multitude of conferences have been held for women's rights and the elimination of discrimination. In the majority of countries in the world, women hold rights equal to men.
\end{enumerate}

\subsubsection{Humans and Ideas}
\label{sec:org6883a4f}
\begin{enumerate}
\item Environmental Consciousness
\label{sec:orge7139cb}

As more data has been gathered, and more people become aware, more attention has been focused towards combating global warming and taking care of the environment.

\item Global Culture
\label{sec:org9d48a00}

Different cultures have spread all around the world. Food, clothing, music, and other elements became shared at a global level.

\item Marxism and Neo-Liberalism
\label{sec:orgfb14f7f}

Two more modern belief systems, Marxism and neo-Liberalism, emerge; however, both have proven to be unpopular.

\item Religion and Science
\label{sec:orga9bad77}

With the globalization of culture, many traditional ideas of religion have been compared and challenged. Religions have adapted to rethink their relationships to other religions, and many local religions have died out. Religious figures and sites have been made much more accessible

In regards to science, massive breakthroughs and research into new fields have emerged. Some, such as biotechnology and genetic engineering, lead to moral dilemmas.

This present Big Era shows radical discovery and advancement, but also leaves many questions moving forward.
\end{enumerate}

\subsubsection{Summary}
\label{sec:org8566add}

After the Cold War ended with capitalism rising as the victor, massive changes greatly affected people on a global scale. The population once again boomed, technological and scientific advancements have been made in abundance, and economies have grown and possibly stabilized. A greater concern than ever has been placed on environmental awareness and climate change. The world is now interconnected, and people from all over the world interact. Not much is certain moving forward, but only time will tell.
\section{The 7 Revolutions}
\label{sec:org749aaa0}
\subsection{Population}
\label{sec:orgc18ccc9}

\begin{itemize}
\item Population is accelerating at a very fast rate.
\item Estimated 10 billion people by 2050.
\item Most of the population increase will come from seven developing countries: India, China, Pakistan, Nigeria, Indonesia, Bangladesh, Brazil.
\item Older people in Africa and South Asia hit with HIV/AIDS, skewing the population towards younger people.
\item Developed countries population trending towards older people. This might cause labor problems and change how governments function.
\item More people are moving from rural areas to urban cities or even out of the country. This could hurt agricultural output.
\end{itemize}
\subsection{Resources}
\label{sec:orgd9bd112}

\begin{itemize}
\item Food demand is growing quickly with increasing population.
\item Degradation and environmental and financial problems are causing developing nations to output less, which is troublesome.
\item A water crisis is emerging in developing nations; not many people have reliable access to clean water.
\item Water shortages could come soon
\item Energy demand is increasing, especially in I and C. This means more use of fossil fuels.
\item Fossil fuels heavily damage our environment. Not many countries are doing anything right now.
\end{itemize}
\subsection{Technology}
\label{sec:org8463049}

\begin{itemize}
\item Computers are advancing at an exponential rate.
\item Similar technology, like phones are becoming part of our daily lives.
\item Improvements in biotechnology could allow for genetically modified and improved humans, but this raises ethical questions.
\item Nanotechnology involves the study and creation of technology at very small scales. It is useful for medicine and other fields.
\item Nanotechnology is used in synthetic biology, which is said to be one of the "next biggest things."
\end{itemize}
\subsection{Information}
\label{sec:org781c7e0}

\begin{itemize}
\item The success of economies and companies is shifting away from being based on manufacturing to revolving around knowledge.
\item The internet allows for better connectivity and marketing, which is a benefit for new entrepreneurs, but it can also encourage piracy and cyber-warfare.
\item Information and technology allows for a more level playing field between the developed and developing world.
\item Information technology could push for better workplace conditions and more learning.
\item One movement that is emerging is the open source movement, which pushes for open, free decentralization of information.
\end{itemize}
\subsection{Economic Integration}
\label{sec:orge12564f}

\begin{itemize}
\item Globalization pushes for the integration (reduction of trade barriers between countries) of economies. This has benefited economies of developing nations.
\item This also has motivated countries to join trading blocs like the EU, Southern African Development Community, and others.
\item Global exports have increased.
\item Integration also causes concerns over national identity, heritage, and integration has not benefited everyone.
\item Some countries have placed restrictions on immigration because of rapid integration.
\item G-6 countries are the US, Japan, Germany, the UK, France, and Italy.
\item BRIC countries are Brazil, Russia, India, and China.
\item By 2025, the BRIC countries could reach or even overtake the G-6 countries with their high population and resources.
\item The BRIC countries also might eventually slow down, just like the developed G-6 countries.
\item Some BRIC citizens do not receive benefits from integration.
\item Russia and the others may be at risk. Russia's economy is greatly based on crude oil.
\item Globalization has not completely gotten rid of poverty, and disparities exist.
\item These great disparities can cause social instability.
\item The middle class is expanding, which could cause a movement to end these disparities.
\end{itemize}
\subsection{Conflict}
\label{sec:orga11d0e9}

\begin{itemize}
\item In modern times, conflict is usually found with terrorism, and not between countries.
\item Terrorist groups have evolved into organized groups with resources and capital.
\item Terrorists finance themselves with arms, drug, and human trafficking.
\item As mentioned earlier, disparities in benefits cause social unrest and could be leading factors in encouraging terrorism.
\item Weapons of Mass Destruction (WMD) are also a large threat in the possession of countries such as North Korea or Iran. They are an even greater threat in the possession of non-governmental bodies.
\item People may be able to obtain nuclear weapon materials on the black market and other places.
\item WMD threats come from both developed and developing countries.
\item Armies need to evolve and change to adapt to their adversaries.
\end{itemize}
\subsection{Governance}
\label{sec:org1a58d00}

\begin{itemize}
\item Companies should not just provide goods and services but also serve the public good.
\item Companies would only do this if it were profitable, but the financial crisis may cause them to reconsider.
\item Civil society organizations (CSOs) and nongovernmental organizations (NGOs) aid in providing governance and services.
\item NGOs play a bigger role in developing countries where the government has failed to meet their needs. This makes people turn to NGOs.
\item Corruption is a major issue pushing back humanitarian progress that affects both the developing and developed world.
\item National, state, and international governing bodies are no longer the central, only sources of governance.
\item The UN and NATO need to adapt to keep up with the problems of countries, or they could fall.
\item Private companies, CSOs, and NGOs have emerged to be major forces in worldwide problems.
\item Humanity needs an integrated, collaborative government.
\item Leaders need to think in the longer term, apply knowledge, and work together to save the planet.
\end{itemize}
\end{document}
