\section{Sampling Distributions}

\begin{blackbox}
    \begin{definition}
        A \textbf{sampling distribution} is a distribution of all possible
        values of a statistic found by taking random samples of a population.
    \end{definition}
\end{blackbox}

\begin{blackbox}
    \begin{definition}
        The \textbf{central limit theorem} states that the sampling
        distribution of any data set with a well defined mean and variance
        approaches a normal distribution.
    \end{definition}
\end{blackbox}

I really should check out the 3b1b video on the central limit theorem.

Given some proportion \( p \) and sample size \( n \), the sampling
distribution of sample proportions has a respective mean and variance of
\begin{align*}
    \mu_X &= p & \variance{X} &= \frac{p \left( 1 - p \right)}{n}
.\end{align*}
A general rule of thumb tells us that this distribution will be roughly normal
if both \( np \ge 10 \) and \( n \left( 1 - p \right) \ge 10 \).

The variance of the sampling distribution decreases as the sample size increases according to the following formula:
\[
    \variance{\overline{X}} = \frac{\variance{X}}{n}
,\]
where \( \overline{X} \) represents the sampling distribution of \( X \) and \(
n \) represents the number of samples.
