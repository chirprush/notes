\section{Collecting Data}

When collecting data and planning studies, it is important to identify the
population and the sample groups and interpret how the data acquired from the
sample can be generalized fairly.

Generally when we think of studies, there are a few types that come to mind:
\begin{itemize}
    \item \textbf{Experiment}: These randomly separate people into groups to
        see if there is any statistical relationship between values when giving
        some sort of stimulus to the population.
    \item \textbf{Retrospective Observational Study}: Using past data to analyze and draw conclusions.
    \item \textbf{Sample Survey}: Using current data and surveying to analyze and draw conclusions. This is also an observational study.
    \item \textbf{Propspective Observational Study}: Gathering future data to analyze and draw conclusions.
\end{itemize}

\begin{blackbox}
    \begin{definition}
        An \textbf{explanatory variable} is just an independent variable. A
        \textbf{response variable} is a dependent variable.
    \end{definition}
\end{blackbox}

There are several different types of bias that may slip into the sampling
process that make any conclusions drawn illegitimate:
\begin{itemize}
    \item \textbf{Voluntary Response Sampling}: When you ask people to
        volunteer for something, this may filter out those that don't like
        something, potentially skewing results.

        If a Youtuber asks their viewers to complete a survey on how much they
        like the channel, it's probably going to be biased in favor.

    \item \textbf{Convenience Sampling}: Using samples and data that are the
        easiest to obtain, but not necessarily the most fair.

    \item \textbf{Nonresponse Bias}: When those who don't respond to a survey
        are significantly statistically different from those who do.

    \item \textbf{Undercoverage}: When a group of people are underrepresented.

        This may be the case with say low-income voters not being able to
        access some sort of survey.

    \item \textbf{Response Bias}: When the question is worded in a way that
        moves someone to a certain side, or alternatively when people will
        answer dishonestly in a way that makes them look better.
\end{itemize}

There are also several random sampling techniques:
\begin{itemize}
    \item \textbf{Simple Random Sampling}: Create some bijection between your
        population and a set of numbers and then get a computer or some source
        of randomness to select from these numbers.
    
    \item \textbf{Stratified Sampling}: In order to better ensure a fair
        distribution from different groups in the population, separate the
        population into different layers or strata and sample from each of
        these equally.

    \item \textbf{Clustered Sampling}: Divide the population into generally
        representative groups or clusters (like classrooms) and then choose
        clusters at random. These function as sorts of mini-populations.

    \item \textbf{Systematic Random Sampling}: Start by surveying a random
        person and then skip over \( n \) people and take a survey again and
        repeat this process. Kinda hard to explain but it makes sense.
\end{itemize}

When considering the blocks in experiments, remember that they are not the
treatments but the groups of people.

A couple of key things to remember:
\begin{enumerate}
    \item Only when the sample is random can we necessarily generalize a conclusion of an experiment to a larger population.
    \item Observational studies cannot yield causal relationships.
\end{enumerate}
