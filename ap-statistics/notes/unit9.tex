\section{Inference for Quantitative Data: Slopes}

When doing statistics to see relationships between data, we often take samples
and then plot them. Sometimes, we may suspect a linear relationship, allowing
us to calculate a regression line. Because these are samples though, we can
only estimate the true population regression line. This is a very common
pattern in statistics, and like always we shall now utilize our tools to
construct confidence intervals and hypotheses to make inferences about these
regression line values (usually the slope) just as one would do so for
proportions or means.

The conditions for inference are as follows:
\begin{itemize}
    \item \textbf{Linear}: The true relation between \( y \) and \( x \) for the population is
        linear.
    \item \textbf{Independent}: A common one; we must be confident that our
        samples are independent.
    \item \textbf{Normal}: For every \( x \), the corresponding distribution of
        \( y \) is normal.
    \item \textbf{Equal Variance}: For every \( x \), the corresponding
        distributions of \( y \) all have the exact same variance.
    \item \textbf{Random}: The samples must be random.
\end{itemize}
In all likelihood we will not have to prove that a regression fulfills all of
these, as some are rather tricky to prove.

The confidence interval for linear regressions is
\[
    b \pm t^* \cdot \text{SE}
,\]
where \( b \) is the sample slope obtained, and \( \text{SE} \) denotes the
standard error coefficient of this slope (usually calculated by the magic black
box program :pensive: I hate not knowing how these are derived). The degrees of
freedom we shall use for \( t^* \) are two less than the number of data points
in the sample (once again I hate not knowing how this is derived).

The \( t \) value for a linear regression line is
\[
    t = \frac{b - \beta}{\text{SE}}
,\]
where \( \beta \) represents the true population slope (which we usually assume
to be \( 0 \) as our null hypothesis).

Note that the for the computer output of a least-squares regression, the
calculated \( p \)-value will be two-sided.
